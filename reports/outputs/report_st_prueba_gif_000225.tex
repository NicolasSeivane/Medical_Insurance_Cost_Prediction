\documentclass[11pt]{article}

\usepackage[utf8]{inputenc}
\usepackage{graphicx}
\usepackage{float}
\usepackage{geometry}
\geometry{margin=2.5cm}

\title{Medical Insurance Cost Prediction}
\author{End-to-End Machine Learning Pipeline}
\date{ 2026-01-03 }

\begin{document}
\maketitle

% =========================
% Problem Overview
% =========================
\section*{Problem Overview}

This report presents the results of an end-to-end machine learning pipeline
developed to predict medical insurance costs based on personal and geographic
attributes. The solution includes data ingestion, model training, evaluation,
scoring, and automated reporting.

% =========================
% Dataset Overview
% =========================
\section*{Dataset Overview}

\begin{itemize}
    \item Number of samples: 1338
    \item Target variable: Medical insurance charges
    \item Features include demographic, health, and regional attributes
\end{itemize}

% =========================
% Model Information
% =========================
\section*{Model Information}

\begin{itemize}
    \item Model type: prueba_gif
    \item Random state: 42
    \item Training strategy: supervised regression
\end{itemize}

% =========================
% Training Metrics
% =========================
\section*{Training Metrics}

\begin{figure}[H]
\centering
\includegraphics[width=0.9\textwidth]{ ../streamlit_figures/training_results_prueba_gif.png }
\caption{Training metrics performance}
\end{figure}

% =========================
% Validation Metrics
% =========================
\section*{Validation Metrics}

\begin{figure}[H]
\centering
\includegraphics[width=0.9\textwidth]{ ../streamlit_figures/scoring_results_prueba_gif.png }
\caption{Validation metrics performance}
\end{figure}

% =========================
% Predictions vs Actual. Solo agrego .png, es mas facil
% =========================
\section*{Predictions vs Actual Values}

\begin{figure}[H]
\centering
\includegraphics[width=0.85\textwidth]{ ../streamlit_figures/scoring_comparation_prueba_gif.png }
\caption{Comparison between real and predicted insurance charges}
\end{figure}

% =========================
% Final Evaluation
% =========================
\section*{Final Evaluation}

The trained model was evaluated on a hold-out dataset generated through random
sampling. The results demonstrate the model's ability to capture the underlying
patterns in medical insurance costs while maintaining generalization performance.

\end{document}